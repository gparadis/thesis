\setcounter{secnumdepth}{0} 

\chapter*{Introduction}
\phantomsection\addcontentsline{toc}{chapter}{Introduction}

This introductory chapter provides some background information on key topics and describes the motivation for this research project. The final section describes the structure of the remainder of this thesis.

\section{Background}

This thesis is about forest management. More specifically, this thesis looks closely at the use of wood supply models by government as a basis for allocating industrial timber licences on public forest land in Canada.
We describe the state of the art in wood supply planning, describe circumstances under which the standard wood supply models may fail to support sustainable forest management, and propose extensions to the standard model that reduce the risk of future wood supply failures.

This section provides some background information, relating key topics in forest management, game theory, and operations research. To introduce these topics, and explain their relationship to each other, let us work backwards from the main contribution of this research project. 

\subsection{Hierarchical Forest Management Planning}

The main contribution of this research is a bilevel extension to the classic wood supply optimization model. The need for such an extension, and justification for the bilevel modelling approach, stems from the distributed nature of the wood supply planning problem on public land. 
%First, we look at the challenges of implementing a hierarchical planning paradigm in a distributed decision-making context. 

In Canada, and in other jurisdictions, government stewards manage public forest on behalf of the people, who ultimately own the land. Civil servants are responsible for a certain upstream portion of the planning process, which they eventually hand off to downstream industrial fibre consumers. The decoupling point between government and industry, although subject to variation between provincial jurisdictions, is typically closely linked to the timber licence contract that binds both parties. 

Thus, we can describe the forest management on public forest land as a distributed planning problem; more specifically, it is an instance of the \emph{principal-agent} problem, which has its roots in game theory. From an operations research perspective, bilevel programming subsumes the principal-agent problem, hence the bilevel formulation of our improved wood supply model. 

In 1995, the Canadian Council of Forest Ministers (CCFM) adopted a framework of six criteria and indicators to guide the implementation of sustainable forest management (SFM) on public land in Canada (cite CCFM 1995, 2003). In summary, these indicators relate to (1) biodiversity, (2) ecosystem condition and productivity, (3) soil and water, (4) role in global ecological cycles, (5) economic and social benefits, and (6) social responsibility.

In many jurisdictions, sustainable forest management is implemented using a hierarchical planning framework (cite  tittler2001hierarchical). Although it is common in forest management, the notion of hierarchical planning was first formally developed in a manufacturing context \citep{hax1977design}. Hierarchical planning necessarily includes both coherent linkages to lower levels and effective feedback loops to higher levels. In its original manufacturing context, hierarchical planning was typically implemented within a centralized decision-making environment. Under centralized hierarchical planning, there is a clear incentive to implement proper linkages and feedback loops. In a distributed decision-making environment (i.e., where independent agents with imperfectly aligned goals are responsible for difference levels of the planning hierarchy), these incentives may be weakened or absent, as we will see is the case of hierarchical forest management planning on public land. 

Despite the widespread and long-standing use of hierarchical planning as a conceptual framework for forest management planning, linkages between long- and short-term forest management planning remain weak, and adaptative feedback loops are essentially non-existent. These deficiencies may compromise long-term sustainability, as we will discuss in Chapter 1. As we mentioned earlier, these weak linkages are related to the distributed nature of the decision-making environment. Furthermore, the government's effective (political) planning horizon is much shorter than its nominal (scientific) planning horizon, which may further compromise the rational deployment of the hierarchical planning paradigm in the context of sustainable forest management on public land. We can gain further insight into the these these issues by analyzing the relationships that bind these actors from a game-theoretic perspective. 

\subsection{Agency Theory}

Thus far, we have presented three actors involved in SFM, which we can (simplistically, for the sake of discussion) refer to as: \emph{public} (i.e., owners of the public forest land), \emph{government} (i.e., stewards with legal mandate to sustainably manage the forest resource on behalf of the public), and \emph{industy} (i.e., consumers of fibre from public forest). We principal-agent theory \citep{laffont2002theory} to describe the relationship between these three actors, and help predict future behaviour under a range of hypothetical scenarios.

Agency theory is a branch of game theory that studies the relationship between two parties, where one party (the \emph{principal}) contracts a second party (the \emph{agent}) to carry out some task on his behalf. The \emph{principal-agent problem} occurs in the presence of \emph{imperfectly aligned interests} and \emph{information asymmetry}. Imperfectly aligned interests are self-explanatory, and ubiquitous (i.e., interests are almost never perfectly aligned). Information asymmetry refers to the impossibility of the principal completely observing the agent's actions. The agent, whom we assume is self-interested and rational, will always exploit the information asymmetry to further his interests.  

The principal-agent problem aptly describes the relationship between the public (principal) and the government (agent) in a forest management context. The public has delegated the responsibility of implementing sustainable forest management to the government. However, the technical complexity of forest management and limited transparency of government activities make it impossible for the principal to fully observe the agent. Furthermore, the agent (i.e., government) is at least partly motivated by political factors, which would provide him with the incentive to compromise long-term sustainability in favour of short-term political gain. For example, the agent has an aversion to announcing major wood supply cuts near the end of an electoral cycle, as this sort of bad news typically has a negative impact on voter perception of the political party in power. The principal has no obvious means of detecting that the agent compromising his sustainable forest management duties because of partisan political interests. 

The principal-agent problem also aptly describes the relationship between government (principal) and industry (agent). Note that in this case, we assume that the government is acting as perfect agent on behalf of the people (i.e., we are ignoring the political component described earlier, and assuming that the government is purely motivated by its duty to sustainably manage the public forest on behalf of the public). Hereinafter, the reader may assume this second principal-agent interpretation (i.e., the government as principal, and industrial fibre consumer as agent), unless otherwise specified.

The principal is responsible for implementing sustainable forest management, which he does by setting policies. However, the principal cannot (due to lack of infrastructure) and will not (due to lack of motivation) go so far as to harvest trees and transform the fibre into valuable forest products. This is why he contracts the agent. The contract binding the agent to the principal essentially sets an upper bound on the volume of fibre the agent may extract from the forest (i.e., annual allowable cut, or AAC), along with a set of rules constraining the execution of harvesting activities. 

AAC is determined by the principal using a complex model (hereinafter referred to the \emph{wood supply model}), that simulates an alternating sequence of harvesting activity and forest growth. Time is typically discretized into planning periods, which typically correspond the established rolling-horizon re-planning cycle length (e.g., 5-year planning periods are common). Wood supply models therefore output a vector of future forest states. 

The wood supply model, which is typically deterministic, can \emph{only} be relied upon to output an accurate vector of future forest states if both future harvesting activity \emph{and} future forest growth are known with certainty. For obvious reasons, this condition is never respected in practice. It is not possible to anticipate wood supply model response to all possible combinations of input data error, on a theoretical basis. 

It is, however, possible to empirically test model sensitivity to input data error, although this is a complex and time-consuming task. There is a growing body of research examining the impact of various sources of forest growth model uncertainty (including natural disturbances) on deterministic wood supply model output (cite stuff here). However, very little attention has been paid to the way we model the sequence of future harvesting activity. 

The wood supply model implicitly assumes that the location, timing, and nature of all future harvesting activity are known with certainty. Harvesting treatments affect the state of the residual forest, which in turn affects growth for the upcoming period, which affects the area available for harvest in the the upcoming period, etc. Harvest and growth components of the model are linked in an alternating feed-back loop. The effect of a small change to the simulated vector of harvest activity can potentially be amplified after several planning periods. Harvesting treatment location and time are typically modelled as decision variables in the wood supply model. Seeing as the principal has claimed that he is managing forests sustainably on evidence provided by the (deterministic) wood supply model, this implies that the principal believes that the vector of harvest activity output from the wood supply model is a good approximation of true future harvest activity. 

If one compares past AAC and harvest levels, it is easy to verify that the wood supply model is systematically over-estimating the level of harvest activity. Furthermore, the bias is skewed in favour of certain type of fibre. For example, only 80% of softwood AAC and 45% of hardwood AAC were consumed in Canada between 1990 and 2012 (cite ccfm).

This species-skewed negative bias, despite being common knowledge, has been largely ignored by both government and industry for several decades. Furthermore, the classic wood supply model fails to account for the fact that it is embedded in a rolling-horizon re-planning process (i.e., only a species-skewed subset of the first period of the optimal solution is ever implemented). We cannot properly test the sustainability of the wood supply planning process without accounting for both the species-skewed consumption bias and the rolling-horizon re-planning context.

One of the contributions of this research project is a simulation framework that can simulate two-stage interaction between principal and agent (i.e., in the first stage the principal allocates the timber licence, in the second stage the agent consumes a profit-maximizing subset of fibre allocation). Agent behaviour is simulated using the LogiLab modelling platform, developed by the FORAC research consortium. After simulating two-stage principal-agent interaction, our platform simulates rolling the planning horizon forward one period (i.e., simulate one period of forest growth using the yield curves from the wood supply model). This principal-agent-growth sequence can be repeated indefinitely. 

To our knowledge, no two-stage rolling-horizon wood-supply simulation platform has been published to date. A one-stage rolling-horizon wood supply simulation platform is described in Daugherty (1991), however he assumes perfect implementation of the first period solution to the wood supply optimization problem.

We use our two-stage rolling-horizon re-planning framework to simulate that the consumption bias can, in fact, induce future wood supply failures \emph{after several planning cycles}. These wood supply failures are undetectable using the standard wood supply planning process. Simulation results for this component of the research project are presented in Chapter X. Later, we also show that we can mitigate the risk of wood supply failures by more accurately predicting harvest levels in the wood supply model. To achieve this, we must first examine the root cause the problematic consumption bias.

By definition AAC only specify an upper bound on harvest volume. The agent is free to consume any subset of allocation fibre. We assume that agent want to maximize his profit, subject to resource constraints and market conditions. Thus, the difference between AAC and actual fibre consumption represents fibre that is either impossible or unprofitable to consume. Ignoring this important bias compromises the rational nature of the modelling exercise, and, by extension, compromises the credibility of the principal's claim that he is sustainably managing the forest on behalf of the public. In other words, the principal is not currently using all the information at his disposal.

The obvious solution to this problem would be to eliminate the consumption bias in the wood supply model. To achieve this, we would need to somehow \emph{filter out} the undesirable fibre (i.e., the subset of AAC that the agent does not consume), which would imply anticipating agent fibre-consumption behaviour.

\subsection{Bilevel Optimization}

From an operations research perspective, our principal-agent problem can be formulated as an optimal contract design problem. The optimal contract (for the principal) would maximal (species-wise) fibre allocation \emph{that the agent will consume entirely}. Anticipating profit-maximizing agent behaviour implies a second optimization model, which, when embedded within the existing wood supply model, induces a bilevel structure. Even the simplest bilevel problems are known to be NP-hard, and our problem is rather complex. Furthermore, many bilevel problem instances (including our bilevel wood supply problem) feature non-convex solution spaces, which makes them even more difficult to solve to global optimality. 

\section{Motivation}

The motivation for this research project stems from first-hand professional experience. After working as a consulting forester and expert wood supply analyst, for both industrial and government clients, it became clear to me that the forest management planning process on Canadian Crown land was failing to provide credible evidence of sustainability. Watercooler conversations with operations foresters provided annecdotal confirmation that mixedwood stands were systematically under-represented in annual harvest plans, as these tended to yield more low-quality hardwood fibre than the local market could absorb. This form of high-grading is technically permitted, to a certain extent, by current regulations in most jurisdictions. Although the situation is common knowledge amongst both government and company foresters, this topic is rarely discussed, and then only in small groups with hushed tones. 

%In plain terms, here is the situation foresters do not like to discuss: the standard wood supply models (way) over-estimates intensity of harvest activity over time, but it makes the politicians look good and gives the companies valuable flexibility when selecting harvest areas.  
Inasmuch as the current regulatory framework provides valuable short-term flexibility for both government (principal) and industry (agent) planners, there is little interest in hastening the advent of a new generation of wood supply planning models, particulary if the primary beneficiaries of the \emph{status quo} suspect (and they would likely be correct) that tightening up the system in the name of sustainability may well reduce current fibre allocation levels.  

%Many foresters may suspect that the aforementionned  highgrading will eventually interfere with sustainability, but the models predict that all is well, and nobody likes to be the bearer of bad news.  

This thesis is a constructive call to action, and represents our best attempt at clearly identifying shortcomings of the current planning process, providing evidence of risk of wood supply failures if the \emph{status quo} is perpetuated, and proposing constructive pathways to restore sustainability to sustainable forest management. 

\section{Thesis Structure}

This is a thesis by publication. The next three chapters are a collection of articles, which either have been or will be published in scientific journals. 

Chapter 2 introduces the distributed wood supply planning problem, including the mathematical formulations for the classic wood supply optimization model and our agent-anticipation optimization model. We describe our two-stage rolling-horizon re-planning simulation framework, which we use to simulate the effect of the species-skewed fibre consumption bias on wood supply sustainability. We conclude this chapter by conjecturing that the classic wood supply model be improved if it were extended to explicitly anticipate industrial fibre consumption behaviour.

Chapter 3 further explores the source of antagonism between the principal and the agent, presents a mathematical formulation for the bilevel distributed wood supply planning problem. In a best-case scenario, the bilevel model can completely eliminate the problematic fibre consumption bias. Using a counter-example, we prove non-convexity of the bilevel solution space for this problem, and show how convexity can be restored by imposing certain limitations on network topology in the lower-level network. We describe an decomposition-based algorithm for the special case, and present a computational experiment showing that our algorithm can solve the bilevel model to global optimality in less than twice the time required to solve the classic model.

Chapter 4 presents a case study, where we compare performance of classic and bilevel model formulations using our two-stage rolling-horizon re-planning simulation framework. We show than the bilevel model improves stability of long-term wood supply projections. Improved stability using the bilevel model tends to come with short-term reductions in fibre allocation to the agent (relative to the base scenario using the classic model), which we anticipate would be ill-received in practice. We show that these reductions in bilevel AAC are only necessary (in the name of sustainability) if the agent persists in maintaining fibre-consumption capacity that is poorly aligned with potential wood supply. Conversely, we show that bilevel AAC can \emph{increase} beyond base case levels if the agent chooses to align his capacity with potential wood supply. This hints at unexplored potential benefit of improved government-industry collaboration in the context of distributed wood supply planning.

Chapter 5 summarizes conclusions from this research project, and describes several promising directions for further research.

%\bibliographystyle{plainnat}
%\bibliography{phd}  


