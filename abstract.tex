\chapter*{Abstract}
\phantomsection\addcontentsline{toc}{chapter}{Abstract} 

The hierarchical forest management (HFM) planning process on public
land may currently be failing on two
levels\cite{paradis2013risk}. At the top level, HFM may not
be providing credible assurance of long-term sustainability of timber
supply and forest ecosystem integrity. At a lower level, HFM may be
failing to fully realize value-creation potential from
timber-harvesting activities through over-constraining of the harvest
planning problem. These failures can be traced back to unrealistic
assumptions implicitly embedded into long-term wood supply
optimization models, which may explain why this problem has received
little attention in the literature. We
model the hierarchical forest management planning process as a as a
two-phase rolling-horizon iterative \emph{principal-agent} problem,
illustrate failure scenarios of \emph{status quo} planning process,
and propose an improved wood supply model formulation.

The \emph{status quo} long-term wood supply planning model formulation
(ie. conventional even-flow wood supply maximization model) does not
explicitly consider the profit-maximizing behaviour of the \emph{agent},
and instead implicitly assumes complete consumption of wood supply in
every planning period, regardless of fiber type or value creation
potential. We extend the \emph{status quo} wood supply model with
explicit anticipation of \emph{agent} profit-maximization behaviour,
thereby allowing the \emph{principal} to choose a wood supply plan
that has improved likelihood of being entirely consumed in the first
planning period, thus restoring validity of total-consumption
assumption embedded in the long-term model formulation.  We model 
the \emph{principal-agent} relationship as a bilevel bilinear program (BBP),
where the top level (leader) represents the government wood supply
planning process, and the lower level (follower) represents the value
creation network (VCN) timber consumption process. To the best of our
knowledge, no BBP formulation of the long-term wood supply planning
problem has been published to date.

We present an illustrative case study showing that the \emph{status
  quo} hierarchical forest planning process (ie. two-phase
rolling-horizon iterative re-planning) is incoherent and
dysfunctional: it fails to demonstrate long-term sustainability of
government-endorsed short-term harvest levels, fails to reliably meet
industrial fiber demand over time, and exacerbates incoherence between
wood supply and fiber demand over several planning iterations. We
identify systematic differences between planned and actual harvesting
activities as an important factor contributing to this dysfunction,
which manifests itself as either instability in long-term wood supply
or lost short-term value creation opportunity. We show that coherence
between long- and short-term planning solutions can be improved
 by manipulating linkages between planning levels.
