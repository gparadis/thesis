\setcounter{secnumdepth}{0} 

\chapter*{Conclusion}
\phantomsection\addcontentsline{toc}{chapter}{Conclusion}



We will show, in Chapter 2, that the classic wood supply model formulation can be extended to explicitly anticipate industrial fibre consumption behaviour. Using agency theory as a framework, we develop a bilevel wood supply optimization model that can eliminate the gap between planned and actual fibre consumption. 

In Chapter Z, we present a number of scenarios comparing the performance of classic and extended wood supply optimization models after several two-stage rolling-horizon re-planning simulation cycles. We conclude that our bilevel model formulation, if implemented in practice, could reduce the risk of wood supply failures. By extension, this would improve the credibility of the government's claim that they are sustainably managing our forests.

Our bilevel model extends the classic model with an anticipation constraint. It is an inescapable mathematical fact that adding a constraint to an optimization can only have a null (best-case) or negative (worst-case) effect on the objective function value. Ceteris paribus, our bilevel model will tend to prescribe lower AAC volume than the classic model. The short-term reduction in fibre allocation is traded off against long-term improvement in long-term sustainability. The agent would obviously prefer to forego these wood supply cuts, however wood supply model implementation is outside of his. 

As discussed earlier, effective government planning horizon (i.e., the true horizon that government officials use, in private, when setting policies) tends to be closely tied to electoral cycle period (e.g., 5 years). This is in sharp contrast to the nominal government planning horizon, which is dictated by forest growth rates (e.g., 150 years). This is the first principal-agent problem at work (i.e., the public as principal, the government as agent). Thus, potential benefit of using our bilevel model (i.e., long-term improvement in sustainability) may well be lost on the government, whereas the short-term wood supply cuts would pose a real problem (i.e., wood supply cut can be spun by opposition parties as job cuts, thereby negatively impacting future electoral results). 

This incoherence between effective and nominal government planning horizons, which lies at the core of the public-government principal-agent problem, represents a real impediment to voluntary government up-take of our proposed solution. Addressing this problem (i.e., how can the contract between the public and government be modified to mitigate negative impact of incoherence between effective and nominal government planning horizons) is beyond the scope of this project. However, the public-government principal-agent relationship represents a good starting point for further study.

Also, our representation of the wood supply planning cycle as a two-stage process implies that industry is a passive recipient of fibre allocation contract (i.e., that there is no possibility of iterative negotiation between the principal and the agent within a give planning cycle). In reality, it is common knowledge that forest products industry lobby groups have a considerable leverage with government (again, due to electoral cycles and the public-government principal-agent relationship discussed earlier). This further muddies the waters, and represents an additional impediment to voluntary up-take of our proposed solution.  

Although formally addressing these technological up-take challenges is beyond the scope of this project, we do provide some insight in Chapter Z, where we present a simulation scenario showing significant potential for sustainable *increases* fibre allocation when the agent voluntarily chooses to modify his fibre-consumption capacity such that it is better aligned with potential wood supply. Government could leverage this potential slack in sustainable fibre supply to incite the industry adopt more collaborative behaviour, thereby avoiding the conflict of interest between sustainability and partisan politics.  
