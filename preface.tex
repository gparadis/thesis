% L'avant-propos est obligatoire dans le cas d'un mémoire ou d'une thèse avec insertion d'articles. Il contient des renseignements sur les coauteurs de chaque article, des précisions sur le rôle exact de l'étudiant dans la préparation de chacun des articles ainsi que sur son statut d'auteur principal ou non.

% L'avant-propos doit aussi mentionner si les articles ont été soumis pour publication et, s'ils ont été publiés. Il fait également état, le cas échéant, de toute modification apportée aux articles insérés par rapport à ceux qui ont fait l'objet d'une publication. Lors du dépôt pour évaluation, la Faculté des études supérieures et postdoctorales refusera un mémoire ou une thèse qui ne comporte pas un avant-propos faisant clairement état de la contribution de l'étudiant à chacun des articles inclus dans le mémoire ou la thèse.

\chapter*{Preface}
\phantomsection\addcontentsline{toc}{chapter}{Preface}

This thesis is submitted in partial fulfillment of the requirements for the degree of \emph{philosophi\ae doctor} (PhD) in forest science at Université Laval. 
The work prensented here was directed by Professor Luc Lebel, co-directed by Professor Sophie D'Amours, and realized in collaboration with Dr. Mathieu Bouchard.

This document has been prepared as an article insertion thesis, and includes three journal articles for which I have acted as the principle researcher and am the principal author. 
My contribution to these papers includes the problem definition, literature review, expermental design, mathematical modelling, expermentation, validation, and writing of the manuscripts. 
Contribution of co-authours to these papers was mostly limited to validation of concepts and review of drafts of each manuscript.
Mathieu Bouchard contributed significantly to the design, implementation, and testing of the mathematical models presented in these papers.

The first article, entitled \emph{On the risk of systematic drift under incoherent hierarchical forest management planning}, was co-authoured by Luc LeBel, Sophie D'Amours, and Mathieu Bouchard. 
This paper was published in the \emph{Canadian Journal of Forest Research}, 43(5):480--492, 2013. 
The text inserted into this thesis is identical to the published manuscript, although the presentation of tables and figures has been adapted to best fit the layout of this document. 

The second article, entitled \emph{A bi-level model formulation for the long-term wood supply problem}, was co-authoured by Mathieu Bouchard, Luc LeBel, and Sophie D'Amours. 
This paper was submitted to the \emph{Annals of Awesomeness}, in September 2013. 
The text inserted into this thesis is identical to the submitted manuscript, although the presentation of tables and figures has been adapted to best fit the layout of this document.

The third article, entitled \emph{Improving coherence of forest management planning using a principal-agent approach}, was co-authoured by Luc LeBel, Sophie D'Amours, and Mathieu Bouchard. 
This paper was submitted to the \emph{Intergalactic Journal of Optimization}, in October 2013. 
The text inserted into this thesis is identical to the submitted manuscript, although the presentation of tables and figures has been adapted to best fit the layout of this document.
